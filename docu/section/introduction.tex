\section{Einleitung}
\noindent
In cloudnativen Umgebungen gelten hohe Anforderungen an das Monitoring.
Es müssen weitaus komplexere und vernetzte Infrastrukturen überwacht und observiert werden.
Der Microservice-Ansatz im Software-Architektur-Design macht es mit
steigender Anzahl an Diensten stetig schwieriger, eine Applikation 
ganzheitlich zu monitoren.

\subsection{Logs}
Logs werden statt auf einer oder 
einer Handvoll physischen Maschinen hingegen auf bis zu Hunderten
verteilten Systemen geschrieben oder nicht geschrieben, sondern lediglich auf 
die Standardausgabe ausgegeben. Um diesen veränderten Rahmenbedingungen 
und Anforderungen nachzukommen, müssen Monitoring-Systeme angepasst werden.
Cloudnative Monitoring-Tools aggregieren Logs einer Applikation
von beliebig vielen vernetzten Systemen in eine gemeinsame Datenbank.
Somit können diese Logs zum einen persistiert werden. Des Weiteren können die 
zusammengeführten Logs von einer Suchmaschine indiziert werden und
im Anschluss effektiv durchsucht werden. Logs von verschiedenen oder 
verwandten Applikationen können miteinander verglichen werden.
Mit Softwarelösungen wie Elasticsearch \cite{elasticsearchWebsite}
können diese Daten analysiert werden, auch mithilfe von Machine Learning.
Dies kann zum Beispiel eine Analyse auf Anomalien sein.
Elasticsearch bietet noch viele weitere Funktionen im Bereich
Analyse von Big Data.
Logs geben in der Regel detailliertere Informationen über Ressourcen als Metriken.

\subsection{Metriken}
Metriken sind typischerweise als Zahlen dargestellte Daten, die meist in Relation
zur relativen Zeit, welche in einer Time-Series-Datenbank persistiert werden.
Ebenso sind unter Metriken jene Werte zu verstehen, 
die auch traditionelle Monitoring-Systeme wie beispielsweise
Nagios \cite{nagiosOrgWebsite}
verwalten. Das sind Prozessor-Auslastung, Arbeitsspeicher-Auslastung,
Festplattenkapazität, etc.
Aber das vielleicht Wichtigste, was man über Metriken wissen muss, ist der Punkt 
über die Möglichkeit, Metriken über Infrastrukturkomponenten hinweg zu korrelieren. 
Angesichts der komplexen Abhängigkeiten, die heute in IT-Umgebungen üblich sind, 
ist die Fähigkeit, Metriken zu einem Gesamtbild zusammenzufügen, eine große Zeitersparnis.
Mit dem Übergang zu cloudnativen Umgebungen wird dies aufgrund der dynamischen Natur der
Cloud-Infrastruktur und der sich ständig ändernden Beziehungen zwischen dieser
Infrastruktur und den Anwendungen noch wichtiger.
Um also auftretende Fehler oder Unstimmigkeiten bei verteilten Anwendungen
effizient erkennen zu können, kommen heutzutage in cloudnativen Monitoring-Systemen
die sogenannten Traces zum Einsatz.

\subsection{Traces}
Traces sind, wie der Name schon verrät,  
Anwendungs-Trace-Daten, die Informationen über bestimmte Anwendungsvorgänge
"verfolgen". Da heutzutage so viele Anwendungen voneinander abhängig sind, 
umfassen diese Vorgänge in der Regel Sprünge durch mehrere Dienste, sogenannte Spans.
Traces bieten somit einen wichtigen Einblick in den Zustand einer Anwendung 
von Ende zu Ende. Sie konzentrieren sich jedoch ausschließlich auf die Anwendungsebene
und bieten nur begrenzte Einblicke in den Zustand der zugrunde liegenden Infrastruktur.
Selbst wenn Traces gesammelt werden also immer noch Metriken benötigt werden, 
um einen vollständigen Überblick über eine Umgebung zu erhalten. 
Application-Performance-Management-Tools (APM-Tools) speisen Trace-Informationen in einen
zentralisierten Metrikspeicher ein, sodass Traces eine gute Datenquelle für eine anwendungszentrierte Sichtweise darstellen.
Der Bedarf an der Sichtweise, die Traces bieten können, wird in container-basierten
Microservice-Architekturen, die nichts anderes als eine Sammlung von zusammengefügten
Services sind, noch größer. Diese Umgebungen können mit dem sogenannten
verteilten Tracing adressiert werden.

Verteiltes Tracing ist die dritte Säule der Observability und bietet Einblicke in die 
Leistung von Vorgängen über Microservices hinweg. Eine Anwendung kann von mehreren Diensten 
abhängen, von denen jeder seinen eigenen Satz an Metriken und Logs hat. 
Verteiltes Tracing ist eine Möglichkeit, Anfragen zu beobachten, während sie sich 
durch verteilte Cloud-Umgebungen bewegen. In diesen komplexen Systemen zeigen Traces 
Probleme auf, die bei den Beziehungen zwischen einzelnen Diensten auftreten können.

\subsection{Projektumsetzung}
In folgenden dieses Papers wird die Installation, sowie die erste Einrichtung
der open-source Lösungen Elasticsearch und Kibana beschrieben.
Diese werden von dem amerikanischen Unternehmen Elasticsearch B.V.
entwickelt. Diese Kombination der eigenständigen Tools
Elasticsearch und Kibana wird auch ELK Stack genannt.  
Dieser Software-Stack kann mit weiteren von Elasticsearch B.V. entwickelten
Software-Lösungen wie Logstash oder Beats zum Elastic Stack erweitert werden.
Zusammen bildet der Stack ein mächtiges Tool zur cloudnativen
Observability.

Observability bezeichnet dabei die Fähigkeit, den internen Zustand eines Systems zu verstehen, 
indem die von ihm erzeugten Daten wie Logs, Metriken und Traces analysiert werden.




